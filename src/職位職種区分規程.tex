\documentclass[10pt,a4paper,uplatex]{jsarticle}
\usepackage{bm}
\usepackage[dvipdfmx]{graphicx}
\usepackage[truedimen,left=25truemm,right=25truemm,top=25truemm,bottom=25truemm]{geometry}
\usepackage{array}
\usepackage{titlesec}
\usepackage{jpdoc}
%\usepackage{tree}
\usepackage[nofiglist,notablist,nomarkers]{endfloat}

\renewcommand{\figurename}{別図}
\newcommand{\figref}[1]{別図\ref{#1}}
\renewcommand{\tablename}{別表}
\newcommand{\tabref}[1]{別表\ref{#1}}

\titleformat*{\section}{\large\bfseries}
\def\title{職位職種区分規程}

\def\alias#1{\十干{#1}}

\begin{document}
	\newpage
{\centering \Large\bf \title  \vskip 0em}
\vskip 2em

\article{目的}
本規程は、株式会社RICOSの勤務する職員(以下、「職員」という)の職務内容を明らかにし、人事・給与等事務処理の円滑化を図るため職位及び職種を区分し、取り扱うために定めるものとする。

\article{職位の区分及び職責}
会社は組織に基づき、必要により次の職位を置くものとする。
\begin{enumerate}
	\item 執行役員
	\item 本部長
	\item 部長
	\item 主任
\end{enumerate}

\term 各職位における職責は以下とする。
\begin{enumerate}
	\item 執行役員: 執行役員は、代表取締役の命を受けて、取締役の業務を補佐し、取締役の特命事項を担う責任者として、特命事項を執行する上での業務運営及び労務管理にわたる全般的な権限を有する。
	\item 本部長: 本部長は代表取締役の命を受け、所管本部内の業務を統括する責任者として、本部内の業務運営及び労務管理にわたる全般的な権限を有する。
	\item 部長: 部長は本部長の命を受け、所管部内の業務を統轄する責任者として、部内の業務運営及び労務管理にわたる全般的な権限を有する。
	\item 主任: 部長の命を受け、所管部内の業務分掌及びプロジェクトを主導する。
\end{enumerate}

\article{職種の区分及び業務内容}
会社は組織に基づき、必要により次の職種を置くものとする。 
\begin{enumerate}
	\item 研究員
	\item 技術員
	\item 営業員
	\item 事務員
\end{enumerate}

\term 各職位における業務内容は以下とする。なお、各本部内における業務分掌は、別に定める「組織規程」によるものとする。
\begin{enumerate}
	\item 研究員: 研究活動及びその他の基盤研究部内の業務分掌を遂行する。
	\item 技術員: 各部内における技術・システム・ソフトウェア開発活動及びその他開発に係る部内の業務分掌を遂行する。
	\item 営業員: 営業活動及びその他の営業部の業務分掌を遂行する。
	\item 事務員: 庶務及びその他の各部内の業務分掌を遂行する。
\end{enumerate}

\article{職位及び職種の決定}
新たに職員になった者については、その職務内容に応じ前条に定めるところにより代表取締役が職位及び職種を決定する。職位及び職種を変更する場合も同様とする。
\term 付与された職位及び職種に基づき、それぞれの職位及び職務で求められている役割期待に応えなければならない。
\term 本条第1項にて決定されたもののほか、職務の実情に応じて特別の職位を併せて付与することができる。

\article{発令方法}
職位及び職種は、代表取締役発令とする。ただし、代表取締役が特に認めた者については、職位及び職種の発令を省略することができる。

\article{職務権限}
職務権限については、別に定める「職務権限規程」によるものとする。

\vspace{1cm}
\subparagraph{附則}本規程の改廃は、「規程類管理規程」に定める手続きによるものとする。
\subparagraph{附則}本規程は、令和2年7月1日から施行する。
\subparagraph{附則}本規程の改正は、令和4年8月15日から施行する。
\subparagraph{附則}本規程の名称を「職種職名区分規程」から「職位職種区分規程」に変更し、本規程の改正は、令和5年11月1日から施行する。
\subparagraph{附則}本規程の改正は、令和7年6月26日から施行する。

% \begin{table}[h]
% \centering
% \caption{株式会社RICOSにおける職種及び職名の区分}
% \begin{tabular}{|l|l|p{2.2cm}p{3cm}|p{5.5cm}|}
% \hline
% 部署 & 職種 & 職名 & &  職務内容 \\
% \hline
% \multirow{3}{*}{基盤研究部} & \multirow{3}{*}{研究職} & 部長 & Manager & 主任研究員の職務のほか、基盤研究部を統括する。\\
% \cline{3-5}
% & & 主任研究員 & Chief Researcher & 研究員の職務のほか、研究プロジェクトを主導する。\\
% \cline{3-5}
% & & 研究員  & Researcher & 研究活動その他の基礎研究部の事務分掌を遂行する。\\
% \hline
% \multirow{3}{*}{応用開発部} & \multirow{3}{*}{開発職} & 部長 & Manager & 主任技術員の職務のほか、応用開発部を統括する。\\
% \cline{3-5}
% & & 主任技術員 & Chief Engineer & 技術員の職務のほか、開発プロジェクトを主導する。\\
% \cline{3-5}
% & & 技術員  & Engineer & 開発活動その他の応用開発部の事務分掌を遂行する。\\
% \hline
% \multirow{3}{*}{営業部} & \multirow{3}{*}{営業職} & 部長  & Manager &  シニアアカウントマネージャーの職務のほか、営業部を統括する。 \\
% \cline{3-5}
% & & シニアアカウントマネージャー & Senior Account Manager & アカウントマネージャーの職務のほか、複数の営業活動を主導する。\\
% \cline{3-5}
% & & アカウントマネージャー & Account Manager & 営業活動その他の営業部の事務分掌を遂行する。\\
% \hline
% \multirow{2}{*}{総務部} & \multirow{2}{*}{事務職} & 部長  & Manager &  事務員の職務のほか、総務部を統括する。 \\
% \cline{3-5}
% & & 事務員 & Clerk & 庶務その他の総務部の事務分掌を遂行する。\\
% \hline
% \end{tabular}
% \label{tab:syokusyukubun}
% \end{table}

\end{document}